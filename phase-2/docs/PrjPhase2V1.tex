\documentclass[12pt]{article}

\usepackage{float}
\usepackage[font=small]{caption}
\usepackage{graphicx}

\begin{document}

\title{
    Parking Assist System: Phase 2 \\ 
    \large Version 1
}
\author{Jeffin Jacob}
\date{}
\maketitle

\newpage

\section{Hardware Components}
The Parking Assist System Prototype consists of:
\begin{enumerate}
    \item Arduino Nano
    \item HC-SR04 Ultrasonic Sonar Distance Sensor
    \item 5V Buzzer
\end{enumerate}
These 3 components are connected by a half-size breadboard and 8 male to male
jumper wires. Power is provided to the Arduino Nano by an M1 MacBook Air via a
USB-C to Micro-B cable.

\section{Software Components}
The codebase consists of:
\begin{itemize}
    \item \texttt{sensor\_task()}
    \item \texttt{setup()}
\end{itemize}
\textit{Sensor Task} is a FreeRTOS task that pulses the ultrasonic sensor to
determine the relative distance of nearby obstacles. It then updates a state
variable accordingly. \textit{Setup} initializes all tasks and semaphores.

\section{Test Scripts and Patterns}
\subsection{Sensor-Task Test}
\textit{Sensor-Task Test} is responsible for verifying the functionality of
Sensor Task. Sensor-Task Test runs concurrently, alongside Sensor Task, 
prompting the tester, via serial output, to place an obstacle at various 
distances from the sensor, and checking whether the state variable has been 
correctly updated, conveying the result to serial output.

\section{Images}
\begin{figure}[H]
    \centering
    \includegraphics[width=0.8\textwidth]{image.jpeg}
    \caption{Parking Assist System Prototype}
\end{figure}

\section{Observations and Notes}
\subsection{Progress}
The project is 50\% complete. I have a working Sensor Task, which writes the
system's state; now, I need to implement a \textit{Buzzer Task}, which will read
the system's state and sound the buzzer at the appropriate frequency.

\subsection{Setbacks}
A minor setback has been the conversion of the duration of ultrasonic pulses to
distance. Further testing must be done to determine the accuracy of the current
implementation.

\end{document}
