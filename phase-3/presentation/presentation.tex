\documentclass{beamer}

\title{Parking Assist System
\large Presentation}
\author{Jeffin Jacob}
\date{December 7, 2025}

\begin{document}

\frame{\titlepage}

\begin{frame}
    \frametitle{Introduction}
    Parking assist systems are used in motor vehicles to alert drivers of 
    obstacles while reversing. They are a hard real-time embedded system because
    missing a deadline may lead to catastrophic failure, a collision. My system
    utilizes an ultrasonic sensor and a buzzer to produce a range of audible
    updates, based on its distance from an obstacle.
\end{frame}

\begin{frame}
    \frametitle{Features}
    As previously aforementioned, my parking assist system features an 
    ultrasonic sensor which determines distance from an obstacle and prompts a 
    buzzer to beep at various frequencies. If the obstacle is 10-15 cm from the 
    sensor, the buzzer will beep once per second.  If the obstacle is 5-10 cm 
    from the sensor, the buzzer will beep twice per second.  If the obstacle is 
    0-5 cm from the sensor, the buzzer will continuously buzz.
\end{frame}

\begin{frame}
    \frametitle{Challenges}
    I faced many challenges while developing my parking assist system. One that
    immediately comes to mind is handling erroneous sensor data. The ultrasonic
    sensor would occasionally process inaccurate readings and produce a distance
    of 0 cm, which would immediately put the system in its final state
    where the buzzer is continuously buzzing. To overcome this I decided to
    ignore distances of 0 cm, and not change the system's state when they are
    encountered. This does not compromise the robustness of the system, as, with
    valid sensor readings, a distance of 0 cm is not possible, and,
    additionally, the ultrasonic sensor I used, the HC-SR04, is not rated for
    anything sub 2 cm.
\end{frame}

\begin{frame}
    \frametitle{Updates}
    I would like to add driver intervention to my parking assist system. For
    example, have the system apply the vehicles brakes to prevent a collision. I
    think this is the next logical step for my parking assist system. I would
    also like to add another sensor to the system because, when it comes to
    safety critical systems, redundancy is key.
\end{frame}

\begin{frame}
    \frametitle{Conclusion}
    I learned a great deal while developing my parking assist system, as it was 
    my first time developing real-time software. I have gained exposure to 
    priority-based tasks, concurrency, and software timers. This experience was 
    invaluable to me and it certainly will not be my last time engineering 
    embedded systems.
\end{frame}

\end{document}
